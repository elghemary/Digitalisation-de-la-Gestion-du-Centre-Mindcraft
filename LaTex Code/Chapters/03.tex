\chapter{Analyse Fonctionnelle}

L’analyse fonctionnelle a pour but d’identifier et de décrire les besoins du centre Mindcraft afin de concevoir une application cohérente, simple d’utilisation et conforme à la réalité du terrain.
Elle vise à traduire les attentes des utilisateurs en fonctions concrètes que le système devra offrir, tout en respectant les contraintes techniques et organisationnelles du centre.

\section{Périmètre fonctionnel}
Le système développé couvre les besoins essentiels de gestion au sein d’un centre de formation. Il permet notamment :
\begin{itemize}
    \item La gestion des élèves (création, mise à jour et consultation).
    \item La gestion des coachs et de leurs informations associées.
    \item La gestion des programmes pédagogiques proposés.
    \item La gestion des séances (planning, coach assigné, capacité…).
    \item La gestion des inscriptions aux programmes / séances.
    \item L’enregistrement et le suivi des paiements.
    \item La génération de rapports et d’états imprimables.
    \item La production d’attestations de participation.
    \item L’accès à un menu principal facilitant la navigation entre les différentes composantes.   
\end{itemize}

\section{Acteurs du système}
Le système comporte deux acteurs principaux qui interagissent avec l’application :
\subsection*{Administrateur}
C’est l’utilisateur principal de l’application. Il peut :
\begin{itemize}
    \item gérer les élèves, coachs, programmes et séances ;
    \item inscrire des élèves dans des séances ;
    \item enregistrer des paiements et éditer des reçus ;
    \item générer des rapports variés (paiements, participations, tableau de bord) ;
    \item mettre à jour les données ou corriger les informations existantes.
\end{itemize}

\subsection*{Coach}
Dans cette version, le coach dispose d’un rôle informatif. Il peut :
\begin{itemize}
    \item consulter les séances auxquelles il est assigné ;
    \item vérifier la liste des élèves inscrits à ses séances ;
    \item consulter le programme associé à sa mission.
\end{itemize}

Chaque fonctionnalité est conçue pour répondre à un besoin réel observé dans la gestion quotidienne du centre.

\begin{itemize}
    \item \textbf{Réduire la charge administrative} en automatisant la saisie et la mise à jour des informations ;
    \item \textbf{Assurer la cohérence des données} grâce à l’intégrité référentielle et à la normalisation des tables ;
    \item \textbf{Améliorer la visibilité} des activités du centre à travers des rapports synthétiques et précis ;
    \item \textbf{Simplifier l’accès à l’information} via une interface claire et une navigation intuitive.
\end{itemize}

\section{Cas d’utilisation}
Les cas d’utilisation représentent les interactions principales entre les acteurs et le système.
Les fonctionnalités essentielles se déclinent comme suit :
\subsection*{Use Case 1: Gérer Élève}
\begin{itemize}
    \item Ajouter un nouvel élève.
    \item Modifier les informations d’un élève existant.
    \item Rechercher et consulter une fiche élève.
\end{itemize}

\subsection*{Use Case 2: Gérer Coach}
\begin{itemize}
    \item Ajouter un coach.
    \item Modifier ses données.
    \item Afficher les séances animées.
\end{itemize}

\subsection*{Use Case 3: Gérer Programme}
\begin{itemize}
    \item Définir les programmes proposés.
    \item Associer des séances à un programme.
\end{itemize}

\subsection*{Use Case 4: Planifier une Séance}
\begin{itemize}
    \item Créer une séance (date, horaire, coach, programme).
    \item Définir la capacité maximale.
\end{itemize}
\subsection*{Use Case 5: Inscrire un Élève}
\begin{itemize}
    \item Sélectionner un élève.
    \item Sélectionner une séance.
    \item Vérifier la capacité.
    \item Enregistrer l’inscription.
\end{itemize}
\subsection*{Use Case 6: Enregistrer Paiement}
\begin{itemize}
    \item Choisir un élève.
    \item Saisir un montant et une période.
    \item Générer un reçu ou un relevé.
\end{itemize}
\subsection*{Use Case 7: Générer Rapports}
\begin{itemize}
    \item Rapport de participation par programme.
    \item Attestation individuelle.
    \item Rapport de paiements mensuels.
    \item Tableau de bord synthétique.
\end{itemize}
Ces cas d’utilisation constituent le cœur fonctionnel de l'application.
