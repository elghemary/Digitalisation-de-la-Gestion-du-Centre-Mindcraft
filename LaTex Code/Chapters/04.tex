\chapter{Conception et Modélisation}

À partir de l’analyse fonctionnelle, la phase de conception a pour objectif de traduire les besoins exprimés en une représentation structurée des données.
Cette étape s’appuie sur la méthode de MCD, qui permet de représenter les objets du système, leurs caractéristiques et leurs relations logiques, avant de passer à l’implémentation dans Microsoft Access.


\section{Modèle Conceptuel de Données}
Le MCD représente les principales entités du système ainsi que les associations qui traduisent leurs liens fonctionnels.
Dans le cadre du centre de formation Mindcraft, les entités identifiées sont les suivantes :


\begin{table}[!htpb]
    \caption{Description des entités principales du système}
    \label{tab:entites}
    \centering
    \renewcommand{\arraystretch}{1.3}
    \setlength{\tabcolsep}{6pt}
    \begin{tabular}{p{2cm} p{5cm} p{8cm}}
        \toprule
        \textbf{Entité} & \textbf{Description} & \textbf{Attributs principaux} \\
        \midrule
        \textbf{Élève} & Représente chaque apprenant inscrit au centre. & ID\_Élève, Nom, Prénom, DateNaissance, NomParent, EmailParent, Téléphone, École \\

        \textbf{Programme} & Regroupe les informations sur les formations proposées. & ID\_Programme, NomProgramme, Description, Catégorie \\

        \textbf{Séance} & Décrit chaque séance programmée. & ID\_Séance, JourSemaine, Heure, DateDébut, DateFin, Capacité\\

        \textbf{Coach} & Représente les formateurs encadrant les séances. & ID\_Coach, Nom, Prénom, Spécialité, TypeRémunération, MontantRémunération, Email, Téléphone \\

        \textbf{Paiement} & Trace les transactions effectuées par les élèves. & ID\_Paiement, Montant, DatePaiement, ModePaiement \\

        \textbf{Salle} & Désigne les salles où se déroulent les séances. & ID\_Salle, NomSalle, CapacitéSalle \\

        \textbf{Attestation} & Représente les certificats délivrés aux élèves. & ID\_Attestation, DateAttestation, Statut, NuméroAttestation \\
        \bottomrule
    \end{tabular}
\end{table}

\subsection*{Associations principales:}

\begin{itemize}
    \item \textbf{Inscription} : relie un élève à une séance (1:N). 
    Un élève peut suivre plusieurs programmes ; un programme regroupe plusieurs élèves.

    \item \textbf{Planification} : chaque séance appartient à un seul programme (1:N entre Programme et Séance).

    \item \textbf{Encadrement} : un coach anime plusieurs séances (1:N entre Coach et Séance).

    \item \textbf{Réservation de salle} : une salle peut accueillir plusieurs séances, mais chaque séance se déroule dans une seule salle (1:N entre Salle et Séance).

    \item \textbf{Règlement} : chaque paiement est lié à un seul élève (1:N entre Élève et Paiement).

    \item \textbf{Attestation} : un élève peut recevoir plusieurs attestations, chacune liée à un programme validé (1:N entre Élève et Attestation).
\end{itemize}

\subsection*{Cardinalités:}

\begin{table}[!htpb]
    \caption{Relations et cardinalités entre les entités principales}
    \label{tab:relations-cardinalites}
    \centering
    \renewcommand{\arraystretch}{1.3}
    \setlength{\tabcolsep}{6pt}
    \begin{tabular}{p{8cm} p{7cm}}
        \toprule
        \textbf{Relation} & \textbf{Cardinalité (min, max)} \\
        \midrule

        % Élève - Inscription
        Élève – Programme (Inscription) 
        & Élève (1,1) ↔ (0,N) Inscription \newline Programme (1,1) ↔ (0,N) Inscription \\

        % Programme - Séance
        Programme – Séance (Planification) 
        & Programme (1,1) ↔ (0,N) Séance \\

        % Coach - Séance
        Coach – Séance (Encadrement) 
        & Coach (1,1) ↔ (0,N) Séance \\

        % Élève - Paiement
        Élève – Paiement (Règlement) 
        & Élève (1,1) ↔ (0,N) Paiement \\

        % Salle - Séance
        Salle – Séance (Occupation / Réservation) 
        & Salle (1,1) ↔ (0,N) Séance \\

        % Élève - Attestation
        Élève – Attestation 
        & Élève (1,1) ↔ (0,N) Attestation \\

        % Programme - Attestation
        Programme – Attestation 
        & Programme (1,1) ↔ (0,N) Attestation \\

        \bottomrule
    \end{tabular}
\end{table}

\begin{figure}[!htpb]
    \centering
    \includegraphics[width=\linewidth]{Figures/MCD.png}
    \caption{MCD de projet}
    \label{fig:figure-01}
\end{figure}


\section{Transformation du MCD → MLD → MPD}
La transformation du MCD en modèle logique (relationnel) consiste à convertir chaque entité et chaque relation en table relationnelle tout en conservant les dépendances fonctionnelles et les contraintes d’intégrité.

\begin{itemize}
    \item Chaque entité devient une table.
    \item Les identifiants deviennent les clés primaires (PK).
    \item Pour une relation 1:N, la clé primaire du côté « 1 » devient une clé étrangère (FK) dans la table du côté « N ».
    \item Pour une relation N:N, une nouvelle table associative est créée, contenant les deux clés primaires et les éventuels attributs de la relation.
\end{itemize}



\begin{table}[!htpb]
    \caption{Description de Modèle logique des données}
    \label{tab:entites}
    \centering
    \renewcommand{\arraystretch}{1.3}
    \setlength{\tabcolsep}{6pt}
    \begin{tabular}{p{2cm} p{5cm} p{8cm}}
        \toprule
        \textbf{Entité} & \textbf{Description} & \textbf{Attributs principaux} \\
        \midrule
        \textbf{Élève} & Représente chaque apprenant inscrit au centre. & ID\_Élève (PK), Nom, Prénom, DateNaissance, NomParent, EmailParent, Téléphone, École \\

        \textbf{Programme} & Regroupe les informations sur les formations proposées. & ID\_Programme (PK), NomProgramme, Description, Catégorie \\

        \textbf{Séance} & Décrit chaque séance programmée. & ID\_Séance (PK), JourSemaine, Heure, DateDébut, DateFin, Capacité, ID\_Programme (FK), ID\_Coach (FK), ID\_Salle (FK) \\

        \textbf{Coach} & Représente les formateurs encadrant les séances. & ID\_Coach (PK), Nom, Prénom, Spécialité, TypeRémunération, MontantRémunération, Email, Téléphone \\

        \textbf{Paiement} & Trace les transactions effectuées par les élèves. & ID\_Paiement (PK), ID\_Élève (FK), Montant, DatePaiement, ModePaiement \\

        \textbf{Salle} & Désigne les salles où se déroulent les séances. & ID\_Salle (PK), NomSalle, CapacitéSalle \\

        \textbf{Attestation} & Représente les certificats délivrés aux élèves. & ID\_Attestation (PK), ID\_Élève (FK), ID\_Programme (FK), DateAttestation, Statut, NuméroAttestation \\

        \bottomrule
    \end{tabular}
\end{table}

\section{Modèle physique de données }
Le MPD correspond à la traduction concrète du modèle logique en tables Access, avec les types de données, les contraintes et les relations matérialisées.
\begin{verbatim}
ÉLÈVE(
    ID_Élève AUTONUM PK,
    Nom TEXT(50),
    Prénom TEXT(50),
    DateNaissance DATE,
    NomParent TEXT(50),
    EmailParent TEXT(100),
    Téléphone TEXT(20),
    École TEXT(50)
)

PROGRAMME(
    ID_Programme AUTONUM PK,
    NomProgramme TEXT(100),
    Description TEXT,
    Catégorie TEXT(50),
    Durée INTEGER
)

COACH(
    ID_Coach AUTONUM PK,
    Nom TEXT(50),
    Prénom TEXT(50),
    Spécialité TEXT(100),
    TypeRémunération TEXT(20),
    MontantRémunération CURRENCY,
    Email TEXT(100),
    Téléphone TEXT(20)
)

SALLE(
    ID_Salle AUTONUM PK,
    NomSalle TEXT(50),
    CapacitéSalle INTEGER,
    Localisation TEXT(50)
)

SÉANCE(
    ID_Séance AUTONUM PK,
    JourSemaine TEXT(20),
    Heure TEXT(10),
    DateDébut DATE,
    DateFin DATE,
    Capacité INTEGER,
    ID_Programme FK,
    ID_Coach FK,
    ID_Salle FK
)

INSCRIPTION(
    ID_Inscription AUTONUM PK,
    DateInscription DATE,
    Statut TEXT(20),
    ID_Élève FK,
    ID_Programme FK
)

PAIEMENT(
    ID_Paiement AUTONUM PK,
    Montant CURRENCY,
    DatePaiement DATE,
    ModePaiement TEXT(20),
    ID_Élève FK
)

ATTESTATION(
    ID_Attestation AUTONUM PK,
    DateAttestation DATE,
    Statut TEXT(20),
    NuméroAttestation TEXT(20),
    ID_Élève FK,
    ID_Programme FK
)
\end{verbatim}

\subsection*{Types de données choisis}

\begin{table}[!htpb]
    \caption{Correspondance entre types conceptuels et types physiques Access}
    \label{tab:types-access}
    \centering
    \renewcommand{\arraystretch}{1.3}
    \setlength{\tabcolsep}{8pt}
    \begin{tabular}{p{6cm} p{6cm}}
        \toprule
        \textbf{Type conceptuel} & \textbf{Type Access (MPD)} \\
        \midrule
        Identifiant unique & AUTONUM (AutoNumber) \\
        Texte court & TEXT(n) \\
        Texte long & LONG TEXT \\
        Numérique entier & INTEGER / LONG \\
        Date & DATE/TIME \\
        Valeur monétaire & CURRENCY \\
        Booléen / Statut & YES/NO ou TEXT(20) \\
        \bottomrule
    \end{tabular}
\end{table}


\subsubsection*{Contraintes d’intégrité}
\begin{itemize}
    \item \textbf{Clés primaires (PK)} :  
    chaque table possède une clé primaire de type AUTONUM garantissant l’unicité des enregistrements.
    
    \item \textbf{Clés étrangères (FK)} :  
    les relations Élève–Inscription, Programme–Séance, Coach–Séance, Salle–Séance, Élève–Paiement et Élève/Programme–Attestation
    sont matérialisées par des clés étrangères assurant la cohérence des liens.
    
    \item \textbf{Intégrité référentielle} :  
    activée dans toutes les relations, avec \emph{mise à jour en cascade} pour propager les modifications d'identifiants
    et \emph{interdiction de suppression} lorsqu’un enregistrement enfant existe (pour éviter les données orphelines).
    
    \item \textbf{Contraintes de domaine} :  
    types de données adaptés (dates, montants, longueurs maximales), champs obligatoires pour les attributs critiques.
\end{itemize}

\section{Normalisation et justification}

L’ensemble des tables du MPD respecte la troisième forme normale (3FN), ce qui garantit l’absence de redondance
et améliore la cohérence et la maintenabilité du système.

\begin{itemize}
    \item \textbf{1FN Atomicité} :  
    tous les attributs sont élémentaires (pas de champs multivalués ou composés).
    
    \item \textbf{2FN Dépendances partielles} :  
    aucune table ne possède de clé primaire composite ; tous les attributs dépendent entièrement de la clé.
    
    \item \textbf{3FN Dépendances transitives} :  
    aucun attribut non-clé ne dépend d’un autre attribut non-clé (par ex. un paiement ne contient pas le nom de l’élève).
\end{itemize}

Cette normalisation assure une réduction significative des anomalies de mise à jour, de suppression et d’insertion,
tout en facilitant l’évolution future du système.
