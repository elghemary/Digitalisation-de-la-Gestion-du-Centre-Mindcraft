\chapter{Présentation du Cadre Professionnel}
\label{cp:user-guide}


\section{Présentation générale}
Mindcraft est un centre de STEM basé à Tanger, spécialisé dans l’enseignement de la robotique et du programmation pour les jeunes.
Créé dans une optique d’éducation moderne et pratique, le centre propose des ateliers interactifs et des programmes pédagogiques orientés vers les STEM (Science, Technology, Engineering, Mathematics).
Il s’adresse à un public varié: enfants, adolescents et étudiants, désireux de développer des compétences techniques et créatives adaptées aux besoins du monde numérique.

Au-delà de la simple transmission de connaissances, Mindcraft vise à initier les apprenants à la pensée logique, à la résolution de problèmes et à l’innovation, tout en leur offrant un environnement collaboratif et motivant.
Grâce à ses partenariats avec des écoles, associations et organisations locales, le centre contribue activement à la promotion de la culture scientifique et technologique dans la région de Tanger-Tétouan-Al Hoceima.

\section{Mission et valeurs}

La mission principale de Mindcraft est de rendre les sciences et la technologie accessibles à tous, à travers une approche basée sur la pratique, l’expérimentation et la créativité.
Ses valeurs fondamentales reposent sur :

\begin{itemize}
    \item \textbf{L’apprentissage actif :} apprendre en manipulant et en expérimentant ;
    \item \textbf{L’innovation pédagogique :} utiliser la robotique, la programmation et la conception numérique comme outils éducatifs ;
    \item \textbf{L’inclusion et la curiosité : } encourager la participation de tous, quel que soit le niveau de départ ;
    \item \textbf{L’excellence et la rigueur :} former des jeunes capables de raisonner, d’analyser et de créer.

\end{itemize}
Ces principes guident l’ensemble des programmes et des décisions stratégiques du centre.

\section{Organisation interne}

L’organisation de Mindcraft repose sur une structure simple et fonctionnelle, favorisant la collaboration entre les différents pôles d’activité :

\begin{itemize}
    \item \textbf{Direction du centre :} supervision générale, coordination des formations et partenariats ;
    \item \textbf{Pôle pédagogique :} encadrement, conception des programmes éducatifs, planification des séances ;
    \item \textbf{Pôle administratif :} gestion des inscriptions, suivi des paiements, communication avec les parents;
\end{itemize}

Cette répartition permet d’assurer un fonctionnement fluide et un suivi personnalisé des élèves.

\section{Problématique et opportunité}
Avant la mise en place du système Access, la gestion quotidienne du centre présentait plusieurs difficultés :
\begin{itemize}
    \item Absence d’une base de données centralisée pour regrouper les informations des élèves, des séances et des paiements;
    \item Risques d’erreurs dans les montants payés ou les dates d’échéance;
    \item Difficulté à retrouver rapidement les informations d’un élève ou d’un programme;
    \item Manque de visibilité sur les paiements en attente ou les séances planifiées.
\end{itemize}
Ces constats ont mis en évidence le besoin urgent d’une solution logicielle fiable, capable d’automatiser les tâches répétitives, de structurer les données et d’offrir un accès rapide et sécurisé aux informations.