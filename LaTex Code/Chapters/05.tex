\chapter{Implémentation sous Microsoft Access}

Après la finalisation du modèle physique de données, l’étape suivante consiste à implémenter l’application au sein de Microsoft Access.
Cette phase vise à transformer le schéma conceptuel et logique en un système complètement fonctionnel.
\section{Création des tables et relations}
Toutes les tables du MPD ont été créées en mode « Conception » dans Access.
\begin{table}[H]
    \caption{Résumé des tables principales de la base de données}
    \label{tab:tables-principales}
    \centering
    \renewcommand{\arraystretch}{1.3}
    \setlength{\tabcolsep}{6pt}
    \begin{tabular}{p{2.5cm} p{8cm} p{3.5cm}}
        \toprule
        \textbf{Table} & \textbf{Champs essentiels} & \textbf{Clés} \\
        \midrule
        \textbf{tblÉlève} & ID\_Élève, Nom, Prénom, DateNaissance, NomParent, EmailParent, Téléphone, École & ID\_Élève (PK) \\
        \textbf{tblProgramme} & ID\_Programme, NomProgramme, Description, Catégorie, Durée & ID\_Programme (PK) \\
        \textbf{tblCoach} & ID\_Coach, Nom, Prénom, Spécialité, Rémunération, Email & ID\_Coach (PK) \\
        \textbf{tblSalle} & ID\_Salle, NomSalle, CapacitéSalle, Localisation & ID\_Salle (PK) \\
        \textbf{tblSéance} & ID\_Séance, JourSemaine, Heure, DateDébut, DateFin, Capacité, ID\_Programme, ID\_Coach, ID\_Salle & ID\_Séance (PK), ID\_Programme, ID\_Coach, ID\_Salle (FK) \\
        \textbf{tblInscription} & ID\_Inscription, DateInscription, Statut, ID\_Élève, ID\_Programme & ID\_Inscription (PK), FKs (Élève, Programme) \\
        \textbf{tblPaiement} & ID\_Paiement, Montant, DatePaiement, ModePaiement, ID\_Élève & ID\_Paiement (PK), ID\_Élève (FK) \\
        \textbf{tblAttestation} & ID\_Attestation, DateAttestation, Statut, ID\_Élève, ID\_Programme & ID\_Attestation (PK), FKs (Élève, Programme) \\
        \bottomrule
    \end{tabular}
\end{table}
Chaque table a reçu :
\begin{itemize}
    \item sa clé primaire (type : Autonumération),
    \item les types de données corrects,
    \item les contraintes de validation nécessaires
    \item det les clés étrangères définissant les dépendances entre entités.
\end{itemize}
\subsection*{Schéma des relations Access}
\begin{figure}[H]
    \centering
    \includegraphics[width=15cm]{Figures/relationships.png}
    \caption{Relations entre les tableaux}
\end{figure}

\begin{figure}[!htpb]
    \centering
    \includegraphics[width=\linewidth]{Figures/tables/tbleleve.png}
    \caption{Tableau Elève}
\end{figure}

\noindent
Les autres tables (\textit{Programme}, \textit{Coach}, \textit{Séance}, etc.) sont présentées en annexe~\ref{annex:tables}.

\section{Requêtes SQL}
Les requêtes ont été construites pour :
\begin{itemize}
    \item filtrer et formater les données affichées dans les formulaires (notamment les zones de liste déroulante),
    \item produire des statistiques à partir des données de paiement et de présence,
    \item alimenter les rapports de synthèse et d'attestation.
\end{itemize}

\subsection{Exemples de requêtes importantes}
\subsection*{Requêtes utilisées pour les zones de liste déroulante (ComboBox)}
Afin d'afficher les \textit{noms complets} au lieu des identifiants numériques dans les formulaires (\texttt{Élève}, \texttt{Coach}, \texttt{Programme}, \texttt{Séance}), plusieurs requêtes ont été créées.  
Ces requêtes, telles que \texttt{qryEleveListe}, \texttt{qryCoachListe}, \texttt{qryProgrammeListe} et \texttt{qrySeanceListe}, permettent de rendre les interfaces plus intuitives pour l'utilisateur.

\paragraph{Exemple : Liste des élèves (qryEleveListe)}
\begin{verbatim}
SELECT
    [tblEleve].[ID_Eleve],
    [tblEleve].[Nom] & " " & [tblEleve].[Prénom] AS NomComplet,
    [tblEleve].[Ecole],
    [tblEleve].[Telephone]
FROM
    tblEleve
ORDER BY
    [tblEleve].[Nom],
    [tblEleve].[Prénom];
\end{verbatim}

Ces requêtes de liste sont également reprises en Annexe~\ref{annex:SQL} pour l’ensemble des tables (\textit{Coach}, \textit{Programme}, \textit{Séance}).

\subsection*{Requête de tableau de bord (vue globale)}
Cette requête récupère des statistiques globales du centre : nombre total d’élèves, de programmes, de séances, de coachs, et le revenu total cumulé.

\begin{verbatim}
SELECT
    Max(
        (
            SELECT Count([ID_Eleve]) FROM [tblEleve]
        )
    ) AS TotalÉlèves,
    Max(
        (
            SELECT Count([ID_Programme]) FROM [tblProgramme]
        )
    ) AS TotalProgrammes,
    Max(
        (
            SELECT Count([ID_Seance]) FROM [tblSéance]
        )
    ) AS TotalSéances,
    Max(
        (
            SELECT Count([ID_Coach]) FROM [tblCoach]
        )
    ) AS TotalCoachs,
    Max(
        (
            SELECT Sum([Montant]) FROM [tblPaiement]
        )
    ) AS RevenuTotal
FROM
    tblEleve;
\end{verbatim}

Elle est utilisée pour alimenter le rapport \texttt{rpt\_TableauDeBordMindcraft}.

Toutes les autres requêtes intermédiaires (de filtrage, de recherche, ou d’alimentation des formulaires) sont détaillées en Annexe~\ref{annex:SQL}

\section{Création des formulaires}
Les formulaires constituent l’interface principale de l’utilisateur.
Ils permettent la saisie, la modification et la consultation des données sans avoir à manipuler directement les tables.

\subsection*{Formulaire Élèves}
\begin{figure}[H]
    \centering
    \includegraphics[width=0.9\linewidth]{Figures/formulaire/frmEleve.png}
    \caption{Formulaire Élève}
\end{figure}
Ce formulaire permet la gestion complète des fiches des élèves. 
L’utilisateur peut y ajouter, modifier ou supprimer un enregistrement. 
Les champs principaux comprennent le nom, le prénom, la date de naissance, l’école et le contact du parent. 
Des boutons de navigation facilitent le passage entre les fiches, 
et un message d’alerte s’affiche en cas de champ obligatoire non renseigné.

\subsection*{Formulaire Coach}
Ce formulaire regroupe toutes les informations relatives aux coachs : identité, spécialité, type de rémunération et coordonnées.
Le formulaire affiche également, sous forme de tableau, la liste des séances associées à chaque coach.
Il facilite la mise à jour du personnel encadrant et permet d’assigner ensuite les coachs aux séances correspondantes.
\begin{figure}[H]
    \centering
    \includegraphics[width=0.9\linewidth]{Figures/formulaire/frmFormateur.png}
    \caption{Formulaire Coach}
\end{figure}

\subsection*{Menu principal}
\begin{figure}[H]
    \centering
    \includegraphics[width=0.8\linewidth]{Figures/formulaire/frmMenu.png}
    \caption{Tableau Menu}
\end{figure}
Le menu principal constitue la page d’accueil de l’application Access. 
Il regroupe, sous forme de boutons les accès directs aux différents formulaires, requêtes et rapports.
Il organise la navigation et améliore considérablement l’ergonomie de l’application.
\\
Les autres formulaires (\textit{Programme}, \textit{Coach}, \textit{Séance}, etc.) sont présentées en annexe~\ref{annex:tables}.

\subsection*{Cohérence visuelle}
Tous les formulaires partagent une même charte graphique : 
couleurs harmonisées, polices uniformes et disposition cohérente des boutons. 
Cette uniformité renforce l’identité visuelle de l’application et facilite la prise en main. 
Chaque formulaire a été testé afin d’assurer la stabilité, la rapidité et la simplicité d’utilisation de l’ensemble du système.

\section{Automatisation avec VBA}
L’intégration du langage \textbf{VBA (Visual Basic for Applications)} a joué un rôle essentiel dans 
l’automatisation de nombreuses opérations de l’application Access et dans l’amélioration de 
l’expérience utilisateur. 
Chaque formulaire de gestion (Élève, Coach, Programme, Séance, Inscription, Paiement) 
inclut des procédures VBA permettant la recherche dynamique, la validation des entrées, 
la navigation fluide et la sécurisation des actions critiques.

Les extraits suivants illustrent les mécanismes VBA clés implémentés dans l’application.  

\subsection*{Recherche dynamique dans les formulaires}
Chaque formulaire dispose d’une zone de recherche permettant de filtrer immédiatement les enregistrements 
en fonction du nom ou du prénom saisi par l’utilisateur.
\begin{verbatim}
Private Sub cmdSearch_Click()
    Dim q As String
    q = Nz(Me.txtSearchStudent, "")
    
    If Len(q) = 0 Then
        Me.FilterOn = False
        Exit Sub
    End If

    q = Replace(q, "'", "''")
    Me.Filter = "[Nom] LIKE '*" & q & "*' OR [Prénom] LIKE '*" & q & "*'"
    Me.FilterOn = True
End Sub

Private Sub cmdClear_Click()
    Me.txtSearchStudent = Null
    Me.FilterOn = False
End Sub
\end{verbatim}

\subsection*{ Gestion des enregistrements (ajout, sauvegarde, suppression)}
Les formulaires disposent de commandes permettant :
\begin{itemize}
    \item d’ajouter un nouvel enregistrement,
    \item de sauvegarder les modifications,
    \item de supprimer un enregistrement après confirmation.
\end{itemize}

\begin{verbatim}
Private Sub cmdNew_Click()
    DoCmd.GoToRecord , , acNewRec
    If Not Me.NewRecord Then
        DoCmd.GoToRecord , , acNewRec
    End If
End Sub

Private Sub cmdSave_Click()
    DoCmd.RunCommand acCmdSaveRecord
End Sub

Private Sub cmdDelete_Click()
    If Me.NewRecord Then Exit Sub
    If MsgBox("Delete this student?", vbYesNo + vbQuestion, "Confirm") = vbYes Then
        DoCmd.RunCommand acCmdDeleteRecord
    End If
End Sub
\end{verbatim}

\subsection*{ Validation automatique avant enregistrement}
Avant de sauvegarder un élève ou une modification, plusieurs vérifications 
sont automatiquement appliquées pour garantir la cohérence des données :
\begin{itemize}
    \item champs obligatoires (Nom, Prénom),
    \item date de naissance non future,
    \item adresse e-mail valide.
\end{itemize}

\begin{verbatim}
Private Sub Form_BeforeUpdate(Cancel As Integer)

    If Nz(Me.Nom, "") = "" Then
        MsgBox "Please enter the student's last name.", vbExclamation
        Me.Nom.SetFocus
        Cancel = True
        Exit Sub
    End If

    If Nz(Me.Prénom, "") = "" Then
        MsgBox "Please enter the student's first name.", vbExclamation
        Me.Prénom.SetFocus
        Cancel = True
        Exit Sub
    End If

    If Not IsNull(Me.DateNaissance) Then
        If Me.DateNaissance > Date Then
            MsgBox "Birth date cannot be in the future.", vbExclamation
            Me.DateNaissance.SetFocus
            Cancel = True
            Exit Sub
        End If
    End If

    If Nz(Me.EmailParent, "") <> "" Then
        Dim e As String
        e = Me.EmailParent
        If InStr(e, "@") = 0 Or InStrRev(e, ".") < InStr(e, "@") + 2 Then
            MsgBox "Please enter a valid parent email.", vbExclamation
            Me.EmailParent.SetFocus
            Cancel = True
        End If
    End If

End Sub
\end{verbatim}

\subsection*{ Navigation entre formulaires}
Depuis le menu principal, chaque bouton ouvre un formulaire spécifique pour faciliter la navigation.

\begin{verbatim}
Private Sub cmdEleves_Click()
    DoCmd.OpenForm "frmEleves"
End Sub

Private Sub cmdPaiements_Click()
    DoCmd.OpenForm "frmPaiements"
End Sub
\end{verbatim}

Toutes les autres procédures VBA spécifiques aux formulaires
(\texttt{frmCoach}, \texttt{frmProgramme}, \texttt{frmSéance}, \texttt{frmInscription}, \texttt{frmPaiement})
sont listées intégralement en annexe~\ref{annex:VBA}.


\section{Création des rapports}

Les rapports ont été conçus afin de fournir à la direction du centre une vue d’ensemble claire, 
synthétique et professionnelle des données enregistrées dans la base. 
Ils permettent non seulement de visualiser les informations essentielles, 
mais également de les regrouper, trier et analyser selon différents critères pertinents. 
Chaque rapport peut être imprimé directement ou exporté automatiquement au format PDF
grâce aux procédures VBA intégrées.

\subsection{Tableau de bord}
Ce rapport présente une vue globale du fonctionnement du centre :
nombre total d’élèves, nombre de programmes actifs, nombre de séances planifiées, 
effectif des coachs et revenu total généré.  
Ces indicateurs sont calculés à l’aide de sous-requêtes SQL et affichés sous forme de 
cartes synthétiques pour une consultation rapide par la direction.
\begin{figure}[H]
    \centering
    \includegraphics[width=0.9\linewidth]{Figures/Rapports/rptTableauBoard.png}
    \caption{Tableau de Board}
\end{figure}

\subsection*{Participation par programme}
Ce rapport regroupe, pour chaque programme, plusieurs informations essentielles :
le nombre de séances planifiées, le nombre d’élèves distincts inscrits et 
le revenu total associé.  
Il permet d’évaluer la performance de chaque formation, d’optimiser la planification 
et d’orienter les décisions pédagogiques et financières du centre.
\begin{figure}[H]
    \centering
    \includegraphics[width=0.9\linewidth]{Figures/Rapports/rptProgramme.png}
    \caption{Rapport des Inscriptions par programme}
\end{figure}

\subsection*{Attestation de participation}
Ce rapport génère automatiquement une attestation personnalisée pour chaque élève.
Les données affichées (nom complet, programme suivi, période de participation et nom du coach) 
proviennent d’une requête agrégée dédiée.  
Le certificat peut être imprimé individuellement depuis le formulaire Élève ou en lot 
depuis le menu principal, et peut également être exporté au format PDF.
\begin{figure}[H]
    \centering
    \includegraphics[width=0.9\linewidth]{Figures/Rapports/rptAttestation.png}
    \caption{Rapport Attestation}
\end{figure}

\subsection*{Relevé des paiements par élève}
Ce rapport présente l’historique détaillé des paiements effectués par chaque élève : 
montants versés, période payée, date de règlement, programme concerné et séance associée.  
Les données sont triées chronologiquement, et un total par élève est calculé automatiquement.  
Ce rapport constitue l’outil principal pour le suivi financier individuel et la détection 
des paiements manquants ou incomplets.
\begin{figure}[H]
    \centering
    \includegraphics[width=0.9\linewidth]{Figures/Rapports/rptPaiementParEleve.png}
    \caption{Rapport paiement par élève}
\end{figure}

\subsection*{Résumé des paiements mensuels}
Ce rapport regroupe tous les paiements du centre par mois et calcule :
le nombre total de paiements, le montant cumulé du mois et le montant moyen par élève.  
Il fournit une vue claire de l’évolution des revenus mensuels du centre et facilite 
le suivi comptable, la planification budgétaire et la prise de décision stratégique.
\begin{figure}[H]
    \centering
    \includegraphics[width=0.9\linewidth]{Figures/Rapports/rptPaiementMensuels.png}
    \caption{Rapport Paiement mensuels}
\end{figure}

\subsection*{Emploi du temps hebdomadaire}
Ce rapport présente l’ensemble des séances planifiées au cours de la semaine.  
Les données affichées incluent le jour, l’heure, le programme, le coach responsable 
et la salle assignée lorsqu’elle est renseignée.  
Les séances sont triées chronologiquement (du lundi au samedi) puis par horaire, 
ce qui permet une lecture fluide et intuitive du planning.

\begin{figure}[H]
    \centering
    \includegraphics[width=0.9\linewidth]{Figures/Rapports/rptEmploiDuTemps.png}
    \caption{Rapport Emploi du Temps}
\end{figure}
