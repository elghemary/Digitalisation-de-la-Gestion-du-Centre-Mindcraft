\chapter{Annexe C: Code VBA}
\label{annex:VBA}

\section*{Formulaire Eleve}
\begin{verbatim}
Private Sub cmdSearch_Click()
    Dim q As String
    q = Nz(Me.txtSearchStudent, "")
    
    If Len(q) = 0 Then
        ' No text ? remove filter
        Me.FilterOn = False
        Exit Sub
    End If
    
    ' Escape single quotes in the search text
    q = Replace(q, "'", "''")
    
    ' Filter on Nom or Prenom contains the typed text (case-insensitive)
    Me.Filter = "[Nom] LIKE '*" & q & "*' OR [Prénom] LIKE '*" & q & "*'"
    Me.FilterOn = True
End Sub


Private Sub cmdClear_Click()
    Me.txtSearchStudent = Null
    Me.FilterOn = False
End Sub


Private Sub cmdNew_Click()
    DoCmd.GoToRecord , , acNewRec
    If Not Me.NewRecord Then
        ' Safety: ensure we’re really on a new record
        DoCmd.GoToRecord , , acNewRec
    End If
End Sub


Private Sub cmdSave_Click()
    ' Triggers BeforeUpdate; then saves if validation passes
    DoCmd.RunCommand acCmdSaveRecord
End Sub


Private Sub cmdDelete_Click()
    If Me.NewRecord Then Exit Sub
    If MsgBox("Delete this student?", vbYesNo + vbQuestion, "Confirm") = vbYes Then
        DoCmd.RunCommand acCmdDeleteRecord
    End If
End Sub


Private Sub Form_BeforeUpdate(Cancel As Integer)
    ' Nom and Prenom must not be empty
    If Nz(Me.Nom, "") = "" Then
        MsgBox "Please enter the student's last name.", vbExclamation, "Missing last name"
        Me.Nom.SetFocus
        Cancel = True
        Exit Sub
    End If
    
    If Nz(Me.Prénom, "") = "" Then
        MsgBox "Please enter the student's first 
        name.", vbExclamation, "Missing first name"
        Me.Prénom.SetFocus
        Cancel = True
        Exit Sub
    End If
    
    ' DateNaissance: if filled, must be <= Today
    If Not IsNull(Me.DateNaissance) Then
        If Me.DateNaissance > Date Then
            MsgBox "Birth date cannot be in the future.", vbExclamation, "Invalid date"
            Me.DateNaissance.SetFocus
            Cancel = True
            Exit Sub
        End If
    End If
    
    ' EmailParent: simple check (contains @ and . after)
    If Nz(Me.EmailParent, "") <> "" Then
        Dim e As String
        e = Me.EmailParent
        If InStr(e, "@") = 0 Or InStrRev(e, ".") < InStr(e, "@") + 2 Then
            MsgBox "Please enter a valid parent email.", vbExclamation, "Invalid email"
            Me.EmailParent.SetFocus
            Cancel = True
            Exit Sub
        End If
    End If
End Sub


\end{verbatim}

\section*{Formulaire Coach}
\begin{verbatim}
' ==============================
' ======== SEARCH / CLEAR =====
' ==============================
Private Sub cmdSearch_Click()
    Dim q As String
    q = Nz(Me.txtSearchCoach, "")
    
    If Len(q) = 0 Then
        Me.FilterOn = False
        Exit Sub
    End If
    
    q = Replace(q, "'", "''")
    ' Filtrer par nom, prénom ou spécialité
    Me.Filter = "([Nom] LIKE '*" & q & "*') OR ([Prénom] LIKE '*" & q & "*') OR 
    ([Spécialité] LIKE '*" & q & "*')"
    Me.FilterOn = True
End Sub

Private Sub cmdClear_Click()
    Me.txtSearchCoach = Null
    Me.FilterOn = False
End Sub


' ==============================
' ======== NEW / SAVE / DELETE =
' ==============================
Private Sub cmdNew_Click()
    DoCmd.GoToRecord , , acNewRec
End Sub

Private Sub cmdSave_Click()
    On Error GoTo Err_Save
    If Me.Dirty Then
        DoCmd.RunCommand acCmdSaveRecord
        MsgBox "Coach enregistré avec succès.", vbInformation, "Sauvegarde"
    End If
    Exit Sub
Err_Save:
    MsgBox "Erreur lors de l’enregistrement : " & Err.Description, vbCritical, "Erreur"
End Sub

Private Sub cmdDelete_Click()
    If Me.NewRecord Then Exit Sub
    If MsgBox("Voulez-vous vraiment supprimer ce coach ?", vbYesNo + vbQuestion, 
    "Confirmation") = vbYes Then
        DoCmd.RunCommand acCmdDeleteRecord
        MsgBox "Coach supprimé.", vbInformation
    End If
End Sub


' ==============================
' ======== NAVIGATION ==========
' ==============================
Private Sub cmdPrev_Click()
    On Error Resume Next
    DoCmd.GoToRecord , , acPrevious
End Sub

Private Sub cmdNext_Click()
    On Error Resume Next
    DoCmd.GoToRecord , , acNext
End Sub


' ==============================
' ======== VALIDATION ==========
' ==============================
Private Sub Form_BeforeUpdate(Cancel As Integer)
    ' Vérifier le nom
    If Nz(Me.Nom, "") = "" Then
        MsgBox "Veuillez saisir le nom du coach.", vbExclamation, "Nom requis"
        Me.Nom.SetFocus
        Cancel = True
        Exit Sub
    End If

    ' Vérifier le prénom
    If Nz(Me.Prénom, "") = "" Then
        MsgBox "Veuillez saisir le prénom du coach.", vbExclamation, "Prénom requis"
        Me.Prénom.SetFocus
        Cancel = True
        Exit Sub
    End If

    ' Vérifier l'email
    If Nz(Me.Email, "") <> "" Then
        Dim e As String
        e = Me.Email
        If InStr(e, "@") = 0 Or InStrRev(e, ".") < InStr(e, "@") + 2 Then
            MsgBox "Veuillez saisir une adresse email valide.", vbExclamation, 
            "Email invalide"
            Me.Email.SetFocus
            Cancel = True
            Exit Sub
        End If
    End If

    ' Vérifier le type de rémunération
    If Nz(Me.TypeRémunération, "") = "" Then
        MsgBox "Veuillez indiquer le type de rémunération.", vbExclamation,
        "Champ requis"
        Me.TypeRémunération.SetFocus
        Cancel = True
        Exit Sub
    End If

    ' Vérifier le montant de rémunération
    If Nz(Me.MontantRémuneration, 0) < 0 Then
        MsgBox "Le montant de rémunération ne peut pas être négatif.", vbExclamation,
        "Valeur invalide"
        Me.MontantRémuneration.SetFocus
        Cancel = True
        Exit Sub
    End If
End Sub


\end{verbatim}

\section*{Formulaire Inscription}
\begin{verbatim}
' ==============================
' ======== SEARCH / CLEAR =====
' ==============================
Private Sub cmdSearch_Click()
    Dim q As String
    q = Nz(Me.txtSearchInscription, "")
    
    If Len(q) = 0 Then
        Me.FilterOn = False
        Exit Sub
    End If
    
    q = Replace(q, "'", "''")
    
    ' Filtrer par nom/prénom de l'élève ou par séance (jour ou heure)
    Me.Filter = "([ID_Eleve] IN (SELECT ID_Eleve FROM tblEleve WHERE Nom LIKE '*" & q & "*' OR Prénom LIKE '*" & q 
    & "*')) OR " & _
                "([ID_Séance] IN (SELECT ID_Séance FROM tblSéance WHERE JourSemaine 
                LIKE '*" & q & "*' OR Heure LIKE '*" & q & "*'))"
    Me.FilterOn = True
End Sub

Private Sub cmdClear_Click()
    Me.txtSearchInscription = Null
    Me.FilterOn = False
End Sub


' ==============================
' ======== NEW / SAVE / DELETE =
' ==============================
Private Sub cmdNew_Click()
    DoCmd.GoToRecord , , acNewRec
End Sub

Private Sub cmdSave_Click()
    On Error GoTo Err_Save
    If Me.Dirty Then
        DoCmd.RunCommand acCmdSaveRecord
        MsgBox "Inscription enregistrée avec succès.", vbInformation, "Sauvegarde"
    End If
    Exit Sub
Err_Save:
    MsgBox "Erreur lors de l’enregistrement : " & Err.Description, vbCritical, "Erreur"
End Sub

Private Sub cmdDelete_Click()
    If Me.NewRecord Then Exit Sub
    If MsgBox("Voulez-vous vraiment supprimer cette inscription ?", vbYesNo + vbQuestion,
    "Confirmation") = vbYes Then
        DoCmd.RunCommand acCmdDeleteRecord
        MsgBox "Inscription supprimée.", vbInformation
    End If
End Sub


' ==============================
' ======== NAVIGATION ==========
' ==============================
Private Sub cmdPrev_Click()
    On Error Resume Next
    DoCmd.GoToRecord , , acPrevious
End Sub

Private Sub cmdNext_Click()
    On Error Resume Next
    DoCmd.GoToRecord , , acNext
End Sub


' ==============================
' ======== VALIDATION ==========
' ==============================
Private Sub Form_BeforeUpdate(Cancel As Integer)
    ' Élève requis
    If IsNull(Me.ID_Eleve) Or Me.ID_Eleve = 0 Then
        MsgBox "Veuillez choisir un élève.", vbExclamation, "Élève requis"
        If ControlExists("cboEleve") Then Me.cboEleve.SetFocus
        Cancel = True
        Exit Sub
    End If
    
    ' Séance requise
    If IsNull(Me.ID_Séance) Or Me.ID_Séance = 0 Then
        MsgBox "Veuillez choisir une séance.", vbExclamation, "Séance requise"
        If ControlExists("cboSeance") Then Me.cboSeance.SetFocus
        Cancel = True
        Exit Sub
    End If
    
    ' Montant payé
    If Nz(Me.MontantPayé, 0) < 0 Then
        MsgBox "Le montant payé ne peut pas être négatif.", vbExclamation, "Montant 
        invalide"
        Me.MontantPayé.SetFocus
        Cancel = True
        Exit Sub
    End If
End Sub


' ==============================
' ======== COMBO CASCADE =======
' ==============================
Private Sub cboEleve_AfterUpdate()
    On Error Resume Next
    If ControlExists("cboSeance") Then
        Me.cboSeance.Requery
        Me.cboSeance = Null
    End If
End Sub


' ==============================
' ======== UTILITY ============
' ==============================
' Vérifie si un contrôle existe avant de l’utiliser
Private Function ControlExists(ctrlName As String) As Boolean
    On Error Resume Next
    ControlExists = Not (Me.Controls(ctrlName) Is Nothing)
End Function


\end{verbatim}

\section*{Formulaire Menu}
\begin{verbatim}
Private Sub cmdTableauDeBord_Click()
    DoCmd.OpenReport "rpt_TableauDeBordMindcraft", acViewPreview
End Sub

Private Sub cmdProgrammeParticipation_Click()
    DoCmd.OpenReport "qryB_ProgrammeParticipation", acViewPreview
End Sub

Private Sub cmdAttestation_Click()
    DoCmd.OpenReport "rpt_AttestationParticipation", acViewPreview
End Sub

Private Sub cmdRelevePaiements_Click()
    DoCmd.OpenReport "rpt_RelevePaiementParEleve", acViewPreview
End Sub

Private Sub cmdResumePaiementsMensuels_Click()
    DoCmd.OpenReport "rpt_RésuméPaiementsMensuels", acViewPreview
End Sub


\end{verbatim}

\section*{Formulaire Paiement}
\begin{verbatim}
' ==============================
' ======== SEARCH / CLEAR =====
' ==============================
Private Sub cmdSearch_Click()
    Dim q As String
    q = Nz(Me.txtSearchPaiement, "")
    
    If Len(q) = 0 Then
        Me.FilterOn = False
        Exit Sub
    End If
    
    q = Replace(q, "'", "''")
    
    ' Filtrer par nom/prénom de l'élève ou par séance (jour/heure)
    Me.Filter = "([ID_Eleve] IN (SELECT ID_Eleve FROM tblEleve WHERE Nom LIKE '*" & q 
    & "*' OR Prénom LIKE '*" & q & "*')) OR " & _
                "([ID_Séance] IN (SELECT ID_Séance FROM tblSéance WHERE JourSemaine 
                LIKE '*" & q & "*' OR Heure LIKE '*" & q & "*'))"
    Me.FilterOn = True
End Sub

Private Sub cmdClear_Click()
    Me.txtSearchPaiement = Null
    Me.FilterOn = False
End Sub


' ==============================
' ======== NEW / SAVE / DELETE =
' ==============================
Private Sub cmdNew_Click()
    DoCmd.GoToRecord , , acNewRec
End Sub

Private Sub cmdSave_Click()
    On Error GoTo Err_Save
    If Me.Dirty Then
        DoCmd.RunCommand acCmdSaveRecord
        MsgBox "Paiement enregistré avec succès.", vbInformation, "Sauvegarde"
    End If
    Exit Sub
Err_Save:
    MsgBox "Erreur lors de l’enregistrement : " & Err.Description, vbCritical, "Erreur"
End Sub

Private Sub cmdDelete_Click()
    If Me.NewRecord Then Exit Sub
    If MsgBox("Voulez-vous vraiment supprimer ce paiement ?", vbYesNo + vbQuestion, 
    "Confirmation") = vbYes Then
        DoCmd.RunCommand acCmdDeleteRecord
        MsgBox "Paiement supprimé.", vbInformation
    End If
End Sub


' ==============================
' ======== NAVIGATION ==========
' ==============================
Private Sub cmdPrev_Click()
    On Error Resume Next
    DoCmd.GoToRecord , , acPrevious
End Sub

Private Sub cmdNext_Click()
    On Error Resume Next
    DoCmd.GoToRecord , , acNext
End Sub


' ==============================
' ======== VALIDATION ==========
' ==============================
Private Sub Form_BeforeUpdate(Cancel As Integer)
    ' Élève requis
    If IsNull(Me.ID_Eleve) Or Me.ID_Eleve = 0 Then
        MsgBox "Veuillez choisir un élève.", vbExclamation, "Élève requis"
        If ControlExists("cboEleve") Then Me.cboEleve.SetFocus
        Cancel = True
        Exit Sub
    End If
    
    ' Séance requise
    If IsNull(Me.ID_Séance) Or Me.ID_Séance = 0 Then
        MsgBox "Veuillez choisir une séance.", vbExclamation, "Séance requise"
        If ControlExists("cboSeance") Then Me.cboSeance.SetFocus
        Cancel = True
        Exit Sub
    End If
    
    ' Montant requis
    If Nz(Me.Montant, 0) <= 0 Then
        MsgBox "Veuillez saisir un montant valide.", vbExclamation, "Montant invalide"
        Me.Montant.SetFocus
        Cancel = True
        Exit Sub
    End If
End Sub


' ==============================
' ======== COMBO CASCADE =======
' ==============================
Private Sub cboEleve_AfterUpdate()
    On Error Resume Next
    If ControlExists("cboSeance") Then
        Me.cboSeance.Requery
        Me.cboSeance = Null
    End If
End Sub


' ==============================
' ======== UTILITY ============
' ==============================
' Vérifie si un contrôle existe avant de l'utiliser
Private Function ControlExists(ctrlName As String) As Boolean
    On Error Resume Next
    ControlExists = Not (Me.Controls(ctrlName) Is Nothing)
End Function

\end{verbatim}

\section*{Formulaire Programme}
\begin{verbatim}
' ==============================
' ======== SEARCH / CLEAR =====
' ==============================
Private Sub cmdSearch_Click()
    Dim q As String
    q = Nz(Me.txtSearchProgramme, "")
    
    If Len(q) = 0 Then
        Me.FilterOn = False
        Exit Sub
    End If
    
    q = Replace(q, "'", "''")
    ' Filtrer par nom ou catégorie
    Me.Filter = "([NomProgramme] LIKE '*" & q & "*') OR ([Catégorie] LIKE '*" & q & "*')"
    Me.FilterOn = True
End Sub

Private Sub cmdClear_Click()
    Me.txtSearchProgramme = Null
    Me.FilterOn = False
End Sub


' ==============================
' ======== NEW / SAVE / DELETE =
' ==============================
Private Sub cmdNew_Click()
    DoCmd.GoToRecord , , acNewRec
End Sub

Private Sub cmdSave_Click()
    On Error GoTo Err_Save
    If Me.Dirty Then
        DoCmd.RunCommand acCmdSaveRecord
        MsgBox "Programme enregistré avec succès.", vbInformation, "Sauvegarde"
    End If
    Exit Sub
Err_Save:
    MsgBox "Erreur lors de l’enregistrement : " & Err.Description, vbCritical, "Erreur"
End Sub

Private Sub cmdDelete_Click()
    If Me.NewRecord Then Exit Sub
    If MsgBox("Voulez-vous vraiment supprimer ce programme ?", vbYesNo + vbQuestion, 
    "Confirmation") = vbYes Then
        DoCmd.RunCommand acCmdDeleteRecord
        MsgBox "Programme supprimé.", vbInformation
    End If
End Sub


' ==============================
' ======== NAVIGATION ==========
' ==============================
Private Sub cmdPrev_Click()
    On Error Resume Next
    DoCmd.GoToRecord , , acPrevious
End Sub

Private Sub cmdNext_Click()
    On Error Resume Next
    DoCmd.GoToRecord , , acNext
End Sub


' ==============================
' ======== VALIDATION ==========
' ==============================
Private Sub Form_BeforeUpdate(Cancel As Integer)
    ' Vérifier le nom du programme
    If Nz(Me.NomProgramme, "") = "" Then
        MsgBox "Veuillez saisir le nom du programme.", vbExclamation, "Nom requis"
        Me.NomProgramme.SetFocus
        Cancel = True
        Exit Sub
    End If

    ' Vérifier la catégorie
    ' ?? Access ne gère pas les accents dans les noms de champs VBA ? utiliser le vrai 
    nom de champ
    ' Si ton champ s'appelle [Catégorie] avec accent, renomme-le sans accent dans 
    la table (ex: Categorie)
    If Nz(Me.Catégorie, "") = "" Then
        MsgBox "Veuillez saisir la catégorie du programme.", vbExclamation, "Catégorie 
        requise"
        Me.Catégorie.SetFocus
        Cancel = True
        Exit Sub
    End If
End Sub


\end{verbatim}
\section*{Formulaire Séance}
\begin{verbatim}
' ==============================
' ======== SEARCH / CLEAR =====
' ==============================
Private Sub cmdSearch_Click()
    Dim q As String
    q = Nz(Me.txtSearchSeance, "")
    
    If Len(q) = 0 Then
        Me.FilterOn = False
        Exit Sub
    End If
    
    q = Replace(q, "'", "''")
    
    ' Filter by Jour, Heure, Programme, or Coach
    Me.Filter = "([JourSemaine] LIKE '*" & q & "*') OR " & _
                "([Heure] LIKE '*" & q & "*') OR " & _
                "([ID_Programme] IN (SELECT ID_Programme FROM tblProgramme WHERE 
                NomProgramme LIKE '*" & q & "*')) OR " & _
                "([ID_Coach] IN (SELECT ID_Coach FROM tblCoach WHERE Nom LIKE 
                '*" & q & "*' OR Prénom LIKE '*" & q & "*'))"
    Me.FilterOn = True
End Sub

Private Sub cmdClear_Click()
    Me.txtSearchSeance = Null
    Me.FilterOn = False
End Sub


' ==============================
' ======== NEW / SAVE / DELETE =
' ==============================
Private Sub cmdNew_Click()
    DoCmd.GoToRecord , , acNewRec
End Sub

Private Sub cmdSave_Click()
    If Me.Dirty Then
        DoCmd.RunCommand acCmdSaveRecord
        MsgBox "Séance enregistrée avec succès.", vbInformation, "Sauvegarde"
    End If
End Sub

Private Sub cmdDelete_Click()
    If Me.NewRecord Then Exit Sub
    If MsgBox("Voulez-vous vraiment supprimer cette séance ?", vbYesNo + 
    vbQuestion, "Confirmation") = vbYes Then
        DoCmd.RunCommand acCmdDeleteRecord
        MsgBox "Séance supprimée.", vbInformation
    End If
End Sub


' ==============================
' ======== NAVIGATION ==========
' ==============================
Private Sub cmdPrev_Click()
    On Error Resume Next
    DoCmd.GoToRecord , , acPrevious
End Sub

Private Sub cmdNext_Click()
    On Error Resume Next
    DoCmd.GoToRecord , , acNext
End Sub


' ==============================
' ======== VALIDATION ==========
' ==============================
Private Sub Form_BeforeUpdate(Cancel As Integer)
    ' Vérification du programme
    If IsNull(Me.ID_Programme) Or Me.ID_Programme = 0 Then
        MsgBox "Veuillez choisir un programme.", vbExclamation, "Programme requis"
        Me.cboProgramme.SetFocus
        Cancel = True
        Exit Sub
    End If
    
    ' Vérification du coach
    If IsNull(Me.ID_Coach) Or Me.ID_Coach = 0 Then
        MsgBox "Veuillez choisir un coach.", vbExclamation, "Coach requis"
        Me.cboCoach.SetFocus
        Cancel = True
        Exit Sub
    End If
    
    ' Vérification de la cohérence des dates
    If Not IsNull(Me.DateDébut) And Not IsNull(Me.DateFin) Then
        If Me.DateFin < Me.DateDébut Then
            MsgBox "La date de fin doit être postérieure à la date de début.", 
            vbExclamation, "Dates invalides"
            Me.DateFin.SetFocus
            Cancel = True
            Exit Sub
        End If
    End If
End Sub



\end{verbatim}
