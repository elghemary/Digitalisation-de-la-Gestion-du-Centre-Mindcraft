\chapter{Annexe B: Code SQL}
\label{annex:SQL}

\section{Requête de génération d’attestation de participation}
Cette requête permet d’obtenir une ligne par élève avec le nom du programme, le coach, et la période de formation.  
Elle est utilisée comme source de données du rapport \texttt{rpt\_AttestationParticipation}.

\begin{verbatim}
SELECT
    [tblEleve].[ID_Eleve],
    [tblEleve].[Nom] & " " & [tblEleve].[Prénom] AS NomComplet,
    First ([tblProgramme].[NomProgramme]) AS NomProgramme,
    Min([tblSéance].[DateDébut]) AS DateDébut,
    Max([tblSéance].[DateFin]) AS DateFin,
    First ([tblCoach].[Nom] & " " & [tblCoach].[Prénom]) AS NomCoach
FROM
    (
        (
            (
                tblEleve
                INNER JOIN tblPaiement ON [tblEleve].[ID_Eleve] = [tblPaiement].[ID_Eleve]
            )
            INNER JOIN tblSéance ON [tblPaiement].[ID_Séance] = [tblSéance].[ID_Seance]
        )
        INNER JOIN tblProgramme ON [tblSéance].[ID_Programme] = [tblProgramme].[ID_Programme]
    )
    INNER JOIN tblCoach ON [tblSéance].[ID_Coach] = [tblCoach].[ID_Coach]
GROUP BY
    [tblEleve].[ID_Eleve],
    [tblEleve].[Nom],
    [tblEleve].[Prénom];
\end{verbatim}

\section{Requête de résumé des paiements mensuels}
Cette requête regroupe les paiements par mois et calcule les totaux et moyennes.  
Elle alimente le rapport \texttt{rpt\_RésuméPaiementsMensuels}.

\begin{verbatim}
SELECT
    Format([DatePaiement], "mmmm yyyy") AS Mois,
    Count([ID_Paiement]) AS NbPaiements,
    Sum([Montant]) AS TotalMensuel,
    Round(Avg([Montant]), 2) AS MontantMoyen
FROM
    tblPaiement
GROUP BY
    Format([DatePaiement], "mmmm yyyy")
ORDER BY
    Min([DatePaiement]);
\end{verbatim}

\section{Nombre d'élèves distincts par programme}
Cette requête permet de compter le nombre d'élèves inscrits dans chaque programme, 
en tenant compte des participations distinctes.  
Elle est utilisée dans le rapport \\\texttt{rpt\_ParticipationParProgramme}.

\begin{verbatim}
SELECT
    q.ID_Programme,
    COUNT(*) AS Nb_Élèves
FROM
    (
        SELECT DISTINCT
            pa.ID_Eleve,
            pr.ID_Programme
        FROM
            (
                tblProgramme AS pr
                INNER JOIN tblSéance AS se ON pr.ID_Programme = se.ID_Programme
            )
            INNER JOIN tblPaiement AS pa ON se.ID_Seance = pa.ID_Séance
    ) AS q
GROUP BY
    q.ID_Programme;
\end{verbatim}

\section{Nombre de séances par programme}
Cette requête calcule le nombre total de séances planifiées pour chaque programme.

\begin{verbatim}
SELECT
    ID_Programme,
    COUNT(ID_Seance) AS NbSéances
FROM
    tblSéance
GROUP BY
    ID_Programme;
\end{verbatim}

\section{Synthèse complète par programme}
Cette requête combine plusieurs sous-requêtes (\texttt{NbSéances}, \texttt{NbÉlèves},
\texttt{RevenusParProgramme})  
pour produire un tableau complet : nombre de séances, nombre d'élèves distincts,
et revenus totaux par programme.  
Elle constitue la base du rapport \texttt{rpt\_ParticipationParProgramme}.

\begin{verbatim}
SELECT
    p.NomProgramme,
    s.NbSéances,
    e.Nb_Élèves,
    r.TotalRevenus
FROM
    (
        (
            tblProgramme AS p
            LEFT JOIN qryB_NbSeances_ParProgramme AS s 
                ON p.ID_Programme = s.ID_Programme
        )
        LEFT JOIN qryB_NbElevesDistinct_ParProgramme AS e 
            ON p.ID_Programme = e.ID_Programme
    )
    LEFT JOIN qryB_Revenus_ParProgramme AS r 
        ON p.ID_Programme = r.ID_Programme
ORDER BY
    p.NomProgramme;
\end{verbatim}

\section{Revenus totaux par programme}
Cette requête calcule le revenu total généré par chaque programme.  
Elle est utilisée dans la synthèse \texttt{rpt\_ParticipationParProgramme}.

\begin{verbatim}
SELECT
    pr.ID_Programme,
    Sum(pa.Montant) AS TotalRevenus
FROM
    (
        tblProgramme AS pr
        INNER JOIN tblSéance AS se ON pr.ID_Programme = se.ID_Programme
    )
    INNER JOIN tblPaiement AS pa ON se.ID_Seance = pa.ID_Séance
GROUP BY
    pr.ID_Programme;
\end{verbatim}

\section{Suivi détaillé des paiements par élève}
Cette requête fournit un historique complet des paiements par élève,
avec le programme, la séance, le montant et la période couverte.  
Elle est utilisée dans le rapport \texttt{rpt\_RelevéPaiementsÉlève}.

\begin{verbatim}
SELECT
    tblEleve.Nom & " " & tblEleve.Prénom AS NomÉlève,
    tblProgramme.NomProgramme,
    tblSéance.JourSemaine,
    tblSéance.Heure,
    tblPaiement.Montant,
    tblPaiement.PériodePaiement,
    tblPaiement.DatePaiement
FROM
    (
        (
            (
                tblEleve
                INNER JOIN tblPaiement 
                    ON tblEleve.ID_Eleve = tblPaiement.ID_Eleve
            )
            INNER JOIN tblSéance 
                ON tblPaiement.ID_Séance = tblSéance.ID_Seance
        )
        INNER JOIN tblProgramme 
            ON tblSéance.ID_Programme = tblProgramme.ID_Programme
    )
    LEFT JOIN tblCoach 
        ON tblSéance.ID_Coach = tblCoach.ID_Coach
ORDER BY
    tblEleve.Nom,
    tblProgramme.NomProgramme,
    tblPaiement.DatePaiement;
\end{verbatim}

\section{Emploi du temps hebdomadaire}
Cette requête agrège les séances, programmes, coachs et salles pour générer 
l'emploi du temps imprimable du centre.  
Elle est utilisée par le rapport \texttt{rpt\_EmploiDuTemps}.

\begin{verbatim}
SELECT
    S.JourSemaine,
    S.Heure,
    P.NomProgramme,
    C.Nom & " " & C.Prénom AS Coach,
    Sa.NomSalle
FROM
    (
        (
            tblSéance AS S
            INNER JOIN tblProgramme AS P 
                ON S.ID_Programme = P.ID_Programme
        )
        INNER JOIN tblCoach AS C 
            ON S.ID_Coach = C.ID_Coach
    )
    LEFT JOIN tblSalle AS Sa 
        ON S.ID_Salle = Sa.ID_Salle
ORDER BY
    S.JourSemaine,
    S.Heure;
\end{verbatim}
